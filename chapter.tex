%------------------------------------------------------------------------------
\chapter{Git\index{git}}
\label{sec:share000.git}
%------------------------------------------------------------------------------
\section{Introduction}
%------------------------------------------------------------------------------
\label{sec:share000.git.introduction}

From WikiPedia:
\begin{quote}
Git is a distributed revision control and source code management (SCM) system
with an emphasis on speed. Git was initially designed and developed by Linus
Torvalds for Linux kernel development in 2005. Every Git working directory is
a full-fledged repository with complete history and full version tracking
capabilities, not dependent on network access or a central server.
\end{quote}

\noindent
Git was chosen because it is available on Windows as well as on Linux. Git
shows a lot of flexibility for open source development. And git has a build-in
mechanism for submodules. We will be using submodules for:
\begin{itemize*}
\item interfaces
\item units (implementation)
\item modules (composition of units and/or modules)
\item sources of documents and figures
\item external tools such as (text) compilers and git
\item external libraries and document
\end{itemize*}

\noindent
Some terminology:
\begin{description}
\item[\ix{repository}] A git repository contains, among other things, a set of
      commit objects and a set of references to commit objects, called heads.
      The git repository is stored in the same directory as the project itself,
      in a subdirectory called !.git!. Note that, in some way, commits
      represent a version, and such a commit typically spans multiple files.
      Git does not version files (separately). A checked out commit is version
      wise always consistent, i.e. all working files are of the same version.
\item[\ix{submodule}] A git submodule exists within one repository and refers
      to an other repository. The containing repository only maintains an URL
      and version, so the containing repository does \emph{not} contain (a
      copy of) the commits of the other repository. When a repository has
      multiple submodules, the repository itself is responsible for combining
      appropriate versions of its submodules. This freedom allows us to have
      separate version control over e.g. an interface and one of its
      implementations.
\item[\ix{stem (repository)}] Stem repository is not a git term. In this
      project, a repository that contains submodules is called the stem
      repository. As we don't have submodules within submodules in this project
      (a design decision), a repository is either a submodule or a stem.
\item[\ix{archive}] Archive is not a git term, it is defined specifically for
      this project. An archive is in implementation a flat collection of
      repositories. This collection typically contains submodules and at least
      one stem.
\end{description}

%~~~~~~~~~~~~~~~~~~~~~~~~~~~~~~~~~~~~~~~~~~~~~~~~~~~~~~~~~~~~~~~~~~~~~~~~~~~~~~
%\begin{figure}
%\begin{center}
%\begin{asy}
%// define nodes
%node class_Archive   =uml_class("Archive");
%node class_Repository=uml_class("Repository");
%node class_Stem      =uml_class("Stem");
%node class_Submodule =uml_class("Submodule");
%
%// dock, flush and draw nodes
%real hu=0.5cm;
%real vu=0.5cm;
%// next to each other, center aligned
%node c1=hdock(2hu, class_Stem, class_Submodule);
%// above each other, center at class_Repository
%node c2=vdock(3vu, centerat=0, class_Repository, c1);
%// next to each other, top aligned
%node cc=hdock(2hu, class_Archive, c2);
%// draw classes
%cc @ refresh @ deepdraw;
%
%// draw edges
%(class_Archive--class_Repository)             .style(uml_aggregation).draw();
%(class_Repository--VHVd(2vu)--class_Stem)     .style(uml_inheritance).draw();
%(class_Repository--VHVd(2vu)--class_Submodule).style(uml_inheritance).draw();
%(class_Stem--class_Submodule)                 .style(uml_associate_d).draw();
%
%// label
%label("*",class_Repository^W,NW);
%label("*",class_Stem^E,SE);
%label("*",class_Submodule^W,NW);
%\end{asy}
%\caption{An archive consists of multiple repositories}\label{fig:git.archive}
%\end{center}
%\end{figure}
%~~~~~~~~~~~~~~~~~~~~~~~~~~~~~~~~~~~~~~~~~~~~~~~~~~~~~~~~~~~~~~~~~~~~~~~~~~~~~~

For a good overview and/or a good introduction into git see \cite{Chacon}.
\cite{Chacon} is a book that a kind of covers all. The following sections are
more focused on how git is used in this project. Git can be used in many
different ways, and choices have been made to come to a single way of working.
An example of such a choice is that we will be using submodules.

%------------------------------------------------------------------------------
\section{Getting started; cloning an archive \label{sec:git.start}}
%------------------------------------------------------------------------------

First you need to be able to connect to the ubuntu server.

\begin{enumerate*}
\item click windows start button
\item Paste the following line in the search box and press enter:\\
      !notepad c:\Windows\System32\Drivers\etc\hosts!
\item Add the following line to the hosts file and save the file:\\
      !130.139.165.20 ubuntu!
\end{enumerate*}

Test the network connection with the ubuntu git server:

\begin{enumerate*}
\item Start a dos command prompt and type:
      !ping ubuntu!
\item The connection is OK when you receive replies.
\end{enumerate*}

Set-up your identity for ssh contact with git server:

\begin{enumerate*}
\item Start a dos command prompt and navigate to the directory: !C:\Users\<your-user-id>\!
\item !mkdir .ssh!\\
      Note: for some reason creating directory !.ssh! does not work using windows explorer
\item See files attached to this E-mail, copy the two files to your !.ssh! directory
\item Create in the same !.ssh! directory a file !config! (without extension)
\item Add the following two lines to this file:\\
      !Host ubuntu!
      !IdentityFile ~/.ssh/id_rsa_ubuntu_git_<your_name>_delete_after_hackaton!
\end{enumerate*}

Flying start:

\begin{enumerate*}
\item Create or reuse a (sub)directory !projects! (required)
\item Create or reuse within this directory a subdirectory !flying_start! (required)
\item Unzip the attached zip-file in the !flying_start! subdirectory.
\item Execute all executables; be prepared to enter your full name once (for git commit messages)\\
      Note: the first takes longer because the others don't need to transport the !share000! stem
\item Check the last message in the command prompt; when it is !done!, then you succeeded
\item When the !projects! directory has new subdirectories !share000! and !hackaton-x!, then it succeeded.
\end{enumerate*}

%------------------------------------------------------------------------------
\section{Getting started with some work \label{sec:git.start2}}
%------------------------------------------------------------------------------

You can start a git shell with the !git-bash.bat! in the stem of your archive.
Suppose you want to do some work, then you must first start with:

\begin{enumerate*}
\item !cd <stem>!
\item !./gs checkout master!
\item !./gs fetch!
\item !./gs rebase origin/master!
\item !./gs checkout -b my_work!
\end{enumerate*}

Otherwise you are working in a deteched head state, which is not very suitable for
development. You may choose other names for !my_work! which describe your work better,
but it has to be a single word.

If you wish to change the name later on, you can use the command:

\begin{enumerate*}
\item !gs branch -m <newname>!
\end{enumerate*}

%------------------------------------------------------------------------------
\section{The state of your local repositories}
%------------------------------------------------------------------------------

Three commands are very useful for showing the state of your repositories:

\index{git branch}
\index{git status -s}
\index{git log}

\begin{lstlisting}[language=]
git branch
git status -s
git log
\end{lstlisting}

\noindent
The first command gives an overview of all existing branches. The branch which
is currently active is prefixed with an asterisk. The second command lists all
changes and the state of the file. The last command lists the last commits.

You can change branch with the command:

\begin{lstlisting}[language=]
./gs checkout <existing branch name>
\end{lstlisting}

You can create a copy of your current branch with

\begin{lstlisting}[language=]
./gs checkout -b <new branch name>
\end{lstlisting}

%------------------------------------------------------------------------------
\section{Making changes \label{sec:git.changing}}
%------------------------------------------------------------------------------

First check what branch you are working on; you probably want to work on
!my_work! and not on the !master!. Then you can simply start editing source
files. Git will keep track of the changed source files. You can list them with:

\begin{lstlisting}[language=]
gs status -s
\end{lstlisting}

\noindent
To add files files for the next commit, you will use the command:

\index{git add ...}

\begin{lstlisting}[language=]
git add <file 1> ... <file n>
\end{lstlisting}

\noindent
The actual commit (after having tested the changes) is as follows:

\index{git commit -m "..."}

\begin{lstlisting}[language=]
git commit -m "<The commit message>"
\end{lstlisting}

\noindent
Now this commit is under source control, in you own repository. You can
continue testing and make further changes. When new changes must be added use
the !git add! command again, when the commit needs to be updated with the new
changes, use the command:

\index{git commit --amend}

\begin{lstlisting}[language=]
git commit --amend
\end{lstlisting}

\noindent
Now the combination of all changes is stored in a single commit. This is
important, because now the reviewer is no longer confronted with all former
coding attempts, the reviewer will only see the final combined commit.

Note that (your) changes still only exist in your repository. So now the
changes need to be presented for review.

%------------------------------------------------------------------------------
\section{Push your work to the central archive}
%------------------------------------------------------------------------------

Dit is me nu even te risicovol om dat in een keer op te schrijven. Volgt later ... helaas.

%------------------------------------------------------------------------------
\section{Receiving updates, rebase}
%------------------------------------------------------------------------------

When the central archive has changed you need to synchronize every now and
then. The changes in that central archive may be changes of others, but also
your own changes are included. When you do not have any work in progress, then
you can start all over again as described in \S\ref{sec:git.start}
p\pageref{sec:git.start}. But more likely, you do have work in progress.

So what you need is a utility that reverts your local archive to the version
you started with, then apply all changes from the central archive, and then
apply your local work in progress. This utility is called !rebase!.

But first you need to fetch the current state of the central archive. This can
be done with the command:

\index{git fetch}

\begin{lstlisting}[language=]
./gs fetch
\end{lstlisting}

\noindent
Note that this is one of the very few commands that require you to be connected
to the internet. After a !git fetch!, the latest changes were obtained from the
remote repositories, and these changes are stored in your local git
repositories. But the !git fetch! did not apply any commit yet on any of your
branches. To apply those commits on your current branch, you need to do the
following:

\index{git rebase origin/master}

\begin{lstlisting}[language=]
./gs rebase origin/master
\end{lstlisting}

A rebase may result in merge conflicts. In such a case you must resolve those
conflicts and then continue the rebase. When you expect conflicts (typically on
a repository with work in progress), then you could do a !git rebase! on the
repository instead of a !./gs rebase! on the archive.

In cases of a merge conflict, read the instructions given by git carefully.
!git status -s! will (also) list all files that need to be merged.
Note additionally, that your branch will be 'detached head', until you have
performed a !rebase --continue! or !rebase --abort!.

%------------------------------------------------------------------------------
\section{Set your changes aside for a short while}
%------------------------------------------------------------------------------

You may have made a few changes, but you should have done a rebase before,
or you may have done some changes, but should have done them on another branch.
In these case, at least when you didn't commit your change yet, you can easily
set your changes aside with:

\index{git stash}

\begin{lstlisting}[language=]
git stash
\end{lstlisting}

\noindent
When you perform a !git status -s! you will see no pending changes. The working
files returned to the version of the last commit.

Now you can e.g. perform a rebase or change branch. Then you can apply your
changes again with:

\index{git stash pop}

\begin{lstlisting}[language=]
git stash pop
\end{lstlisting}

%------------------------------------------------------------------------------
\section{Reverting git actions}
%------------------------------------------------------------------------------

%..............................................................................
\subsection{Reverting an edit}

Suppose that you made changes to files, but you want to revert to the version
you started with, for those files. Also suppose that you did not !add! these
files yet. In this case you can undo all your edits with:

\index{git checkout ...}

\begin{lstlisting}[language=]
git checkout <file 1> <file 2> ...
\end{lstlisting}

%..............................................................................
\subsection{Undo add}

Suppose you added a file for a future commit, but you want to undo this !add!.
Use

\index{git reset ...}

\begin{lstlisting}[language=]
git reset <file 1> <file 2> ...
\end{lstlisting}

%..............................................................................
\subsection{Revert, amend or squash a commit}

Suppose you committed, but you want to revert the commit, but keep all the
changes.

\index{git reset HEAD~1}

\begin{lstlisting}[language=]
git reset HEAD~1
\end{lstlisting}

\noindent
Suppose you committed, but you want to git rid of all changes in that commit.
Then use:

\index{git reset --hard HEAD~1}

\begin{lstlisting}[language=]
git reset --hard HEAD~1
\end{lstlisting}

\noindent
Suppose you committed, but you want to make more changes, and add them to the
commit. In this case simply make those changes and !add! them, and then use:

\begin{lstlisting}[language=]
git commit --amend
\end{lstlisting}

\noindent
Note that amending a commit is very common. Often, you need to make some
additional changes after e.g. finding issues after a test or e.g. when you
receive review comments. It is good practice to combine all changes that belong
to a single task. So don't forget the !--amend! option.

Then it is possible that you created multiple commits, but afterwards you
realize that those commits all belong to the same task. In that case you can
combine those commits in a single commit. This is called squashing. Supposing
you want to combine three commits, then use:

\index{git rebase --interactive HEAD~...}

\begin{lstlisting}[language=]
git rebase --interactive HEAD~3
\end{lstlisting}

\noindent
This command will come up with an editor which will list 3 commits. Of commits
2 and 3, change the !pick! into !s!. Then save and close the file. Now commits
2 and 3 are squashed into 1. It is advisable to check the internet about this
command and inform yourself about how this command works, when you do this for
the very first time.

%------------------------------------------------------------------------------
\section{Working with branches}
%------------------------------------------------------------------------------

Once you are used to it, branches are a convenient way to manage features,
issues or aspects. Let's suppose you are working on a 'larger' task, and that
work is to be developed in the branch !my_work!. You can create such a branch
in the following way:

\begin{lstlisting}[language=]
cd <stem>
./gs checkout master
./gs fetch
./gs rebase origin/master
./gs checkout -b my_work
\end{lstlisting}

\noindent
The first three statements will make the !master! branch active, and assure
that you have the latest version. The last statement creates a new branch
called !my_work! with an identical content to the master. By using the !./gs!
command (i.s.o. !git!), for the stem and all submodules a branch is created.

Now suppose you intend to start with some feature. Possibly you decide later on
not to include this feature, or to implement it differently, and therefore you
want to be able apply and revert this attempt later on. Then you can create
you own feature branch:

\begin{lstlisting}[language=]
cd <relevant submodule>
git checkout my_work
git checkout -b my_feature
\end{lstlisting}

\noindent
Now, for a specific submodule a !my_feature! branch is created. You can do your
work and !add! and !commit! as described in \S\ref{sec:git.changing}
p\pageref{sec:git.changing}. After that, you can merge your commit(s) back into
the !my_work! branch:

\begin{lstlisting}[language=]
git checkout my_work
git merge my_feature
\end{lstlisting}

%------------------------------------------------------------------------------
\section{Install Git for Windows\label{sec:git.install.win}}
%------------------------------------------------------------------------------

\begin{enumerate*}
\item Navigate to the "git" directory
\item Right click !Git-1.9.2-preview20140411.exe! and select "run as
      administrator"
\item Next
\item Next
\item Next
\item Folder: !C:\Program Files (x86)\Git!
\item Next
\item unselect all components
\item Next
\item Next
\item Use Git from GitBash only
\item Next
\item Checkout as-is, commit as-is
\item Next
\item Finish
\item Open !Git Bash! (Windows button | all programs | git)
\item type:\\ 
      !git config --global user.name "John Doe"! \\
      !git config --global user.email johndoe@example.com! \\
      !git config --global push.default simple!
\end{enumerate*}

\noindent
As we will be using submodules, the following bash script may be handy:

\index{gs}
\begin{lstlisting}[frame=trBL,
   float,
   language=,
   caption={gs, a git script for submodules as we use it},
   label={lst:git.gs}]
#!/bin/bash
echo @ gs: git submodule foreach git $1 $2 $3 $4 $5 $6 $7 $8 $9
git submodule foreach git $1 $2 $3 $4 $5 $6 $7 $8 $9
echo @ gs: git $1 $2 $3 $4 $5 $6 $7 $8 $9
git $1 $2 $3 $4 $5 $6 $7 $8 $9
\end{lstlisting}

\noindent
Save this script in the stem of your archive, under the name !gs!. 
This script can only be executed in the stem.
It will execute a !git! command with all parameters in every
submodule and after that it will execute !git! in the stem with the same
parameters, e.g.:

\begin{lstlisting}[language=]
./gs status -s
\end{lstlisting}

%------------------------------------------------------------------------------
\section{Install Git for Linux \label{sec:git.install.lin}}
%------------------------------------------------------------------------------

\tbc

%------------------------------------------------------------------------------
\section{How Git for Windows was downloaded }
%------------------------------------------------------------------------------

For the project, it is preferred that every participant uses the same version
of tools. This also applies for Git. Therefore it is not preferred that
participants download their 'own' or the newest version of a tool. Instead an
installation CD (or an ISO image) is prepared, which contains all appropriate
versions. So please use the CD instead of following instructions below. The
instructions below are only included to be able to create a new version of the
installation CD.

\begin{itemize*}
\item Date of download: May \nth{8}, 2014
\end{itemize*}
\begin{enumerate*}
\item went to \url{http://msysgit.github.io/}
\item pressed download
\item The file !Git-1.9.2-preview20140411.exe! was downloaded
\end{enumerate*}

\noindent
The !sendmail! utility was locally developed in C-sharp. The sources are
archived in the repository !sendmail!.

%------------------------------------------------------------------------------
\section{How Git for Linux was downloaded }
%------------------------------------------------------------------------------

\tbc

%------------------------------------------------------------------------------
\section{Cloning an archive without flying start \label{sec:git.startnofly}}
%------------------------------------------------------------------------------

In case you haven't installed git yet, please follow instructions in
\S\ref{sec:git.install.win} or \S\ref{sec:git.install.lin}
from p\pageref{sec:git.install.win} onwards.

\index{git clone ...}
\index{git submodule update ...}
\index{git checkout -b ...}

\begin{enumerate*}
\item Create a !src! directory
\item !cd src!
\item !git clone git@ubuntu:hackaton-1.git .!
\item !git submodule update --rebase --init!
\item !gs checkout -b my_work!
\end{enumerate*}

\index{gs}

\noindent
!./gs! will execute a !git! command with the given parameters in every
submodule and after that it will execute !git! in the stem with the same
parameters. !./gs! is to be executed in the stem.

\noindent
Note, !my_work! is the name of a newly created branch.

%------------------------------------------------------------------------------
\section{How the archive was setup for the first time}
%------------------------------------------------------------------------------

In the project, participants will use an existing archive. But for some
future other project, the reader may need to set up an archive for himself.
Therefor the log below is included about how the archive was setup for the
first time. Also when new repositories are to be added to the archive, a part
of the instruction below are relevant for the archive owner.

First directories were created with source code and documentation. In the
directory structure, the files that need to be under source control were
separated from other files (like intermediate files, targets and tools).

\begin{itemize*}
\item For every submodule in the source tree:
   \begin{itemize*}
   \item !git init! $\Rightarrow$ this creates an empty archive
   \item !git status -s! $\Rightarrow$ will show uncontrolled files
   \item For files that must be added: !git add <filename1> <filename2> ...!
   \item For files that are to be ignored create a file !.gitignore! and add
         the filename to this file
   \item !git add .gitignore! (when there is one)
   \item !git commit -m "<some comments>"!
   \end{itemize*}
\item (first time only) Create an account for github
\item For every submodule in the source tree:
   \begin{itemize*}
   \item Create an empty repository in github
   \item !cd <your local repository>!
   \item !git remote add origin <URL provided by github>!
   \item !git push -u origin master!
   \end{itemize*}
\item Return to the initial source tree and delete all submodules with the
      initially created repositories. Or move them away as backup.
\item (first time only)
   \begin{itemize*}
   \item !git init! $\Rightarrow$ this creates the empty archive that is to
         contain all submodules
   \item !git submodule init! $\Rightarrow$ this creates the administration for
         submodules
   \end{itemize*}
\item For every submodule in the source tree:
   \begin{itemize*}
   \item !git submodule add <URL provided by github> [directory]! $\Rightarrow$
         to add repository to the stem and store it in ![directory]!.
   \end{itemize*}
\item !git commit -m "<some comments>"!
\item (first time only)
   \begin{itemize*}
   \item !git submodule update! $\Rightarrow$ this updates the sources of all
         submodules
   \item Create an empty repository in github
   \item !git remote add origin <URL provided by github>!
   \item !git push -u origin master!
   \item Now you can delete (or backup) the complete original source tree and
         clone it freshly from the newly created archive
   \end{itemize*}
\end{itemize*}

%------------------------------------------------------------------------------
\section{About ssh}
%------------------------------------------------------------------------------

%..............................................................................
\subsection{Creating \ix{ssh key} for github}

Github is configured to use \ix{ssh} (\abbr{s}ecure \abbr{sh}ell)
authentication. Ssh keys serve as a means of
identifying yourself to github using public-key cryptography and 
challenge-response authentication. One immediate advantage this method has over
traditional password authentication is, that you can be authenticated by github
without ever having to send your password over the network.

In the bash shell type the following:

\begin{lstlisting}[language=]
ssh-keygen -t rsa -C "j.p.lammertink@kubicas.com"
eval $(ssh-agent)
ssh-add ~/.ssh/id_rsa
clip < ~/.ssh/id_rsa.pub
\end{lstlisting}

\noindent
The !clip! command will copy !~/.ssh/id_rsa.pub! to your clipboard.

%..............................................................................
\subsection{Typing the passphrase once}

\index{SSH passphrase}

In a single bash session, there is no need to type the passphrase any time ssh
authentication is needed. Instead you can run the !ssh-agent!, which will
remember your passphrase for the entire rest of the bash session.

\begin{lstlisting}[language=]
eval $(ssh-agent)
ssh-add
\end{lstlisting}

\noindent
!ssh-add! will ask for the passphrase.
